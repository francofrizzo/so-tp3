% ALGUNOS PAQUETES REQUERIDOS (EN UBUNTU): %
% ========================================
% %
% texlive-latex-base %
% texlive-latex-recommended %
% texlive-fonts-recommended %
% texlive-latex-extra %
% texlive-lang-spanish (en ubuntu 13.10) %
% ******************************************************** %

\documentclass[a4paper]{article}
\usepackage[spanish]{babel}
\usepackage[utf8]{inputenc}
\usepackage{fancyhdr}
\usepackage[pdftex]{graphicx}
\usepackage{sidecap}
\usepackage{caption}
\usepackage{subcaption}
\usepackage{booktabs}
\usepackage{makeidx}
\usepackage{float}
\usepackage{amsmath, amsthm, amssymb}
\usepackage{amsfonts}
\usepackage{sectsty}
\usepackage{wrapfig}
\usepackage{listings}
\usepackage{pgfplots}
\usepackage{enumitem}
\usepackage{hyperref}
\usepackage{listings}
\usepackage{listingsutf8}

\linespread{factor}

\definecolor{mygreen}{rgb}{0,0.6,0}
\definecolor{mygray}{rgb}{0.5,0.5,0.5}
\pgfplotsset{compat=1.3}
\setlist[enumerate]{label*=\arabic*.}
\lstset{
	inputencoding=utf8/latin1,
	language=C++,
	basicstyle=\ttfamily,
	keywordstyle=\bfseries\color{blue},
	stringstyle=\color{red}\ttfamily,
	commentstyle=\color{mygreen}\ttfamily,
	morecomment=[l][\color{magenta}]{\#},
	numbers=left,
	numberstyle=\color{mygray}
}

\input{codesnippet}
\input{page.layout}
\usepackage{caratula}

\newcommand{\ord}{\ensuremath{\operatorname{O}}}
\newcommand{\nat}{\ensuremath{\mathbb{N}}}
\newcommand{\acr}[1]{\textsc{\lowercase{#1}}}

%\lstset{
%    language=C++,
%    basicstyle=\ttfamily,
%    keywordstyle=\color{blue}\ttfamily,
%    stringstyle=\color{red}\ttfamily,
%    commentstyle=\color{ForestGreen}\ttfamily,
%    morecomment=[l][\color{magenta}]{\#}
%}

\begin{document}
\materia{Sistemas Operativos}
\submateria{Primer Cuatrimestre de 2016}
\titulo{Trabajo Práctico 3}
\subtitulo{Sistemas Distribuidos}
\integrante{Franco Frizzo}{013/14}{francofrizzo@gmail.com}
\integrante{Iván Pondal}{078/14}{ivan.pondal@gmail.com}
\integrante{Maximiliano León Paz}{251/14}{m4xileon@gmail.com}

\maketitle
% no footer on the first page
\thispagestyle{empty}

\newpage
\section{Introducción}

Para este trabajo se requería completar el protocolo de elección de líder en un
sistema distribuido con topología de anillo, utilizando la implementación en
\texttt{C} de \acr{MPI} (Message Passing Interface). El mismo debía poder
ejecutarse teniendo nodos caídos; por lo tanto, era necesario además
desarrollar algún mecanismo que resistiera bajas.

El protocolo consiste en dos llamadas: \texttt{iniciar\_eleccion} y
\texttt{eleccion\_lider}. A continuación se procede a explicar en detalle las
implementaciones de cada una.

\section{Inicio de elección}

Los procesos $P_i$ encargados de iniciar una elección son seleccionados al azar. Los
mismos tienen como tarea enviar un mensaje $(P_i, P_i)$ a su sucesor en el
anillo donde le indican que iniciaron una elección postulándose como
candidatos.

Para lograr esto se utiliza \texttt{MPI\_Isend} para enviar el token descrito
anteriormente al siguiente otorgado por la función \texttt{siguiente\_pid}.
Como el próximo nodo podría estar caído, se debe esperar a recibir un
\acr{ACK} del destinatario. Esto se logra esperando a que
\texttt{MPI\_Iprobe} detecte que llegó un mensaje con el tag
\texttt{TAG\_ELECCION\_ACK} o que finalice el tiempo de espera. Si se detecta un
\acr{ACK} finaliza la rutina de inicio de elección; en caso contrario, es
necesario reenviar el mensaje al próximo en el anillo. Para lograr esto, se
aumenta en uno el valor obtenido por \texttt{siguiente\_pid}. Esto funciona por
el hecho de que en el peor escenario se llega al último elemento que
necesariamente va a estar vivo evitando así irse de rango.

Cabe destacar que se contempla el escenario donde el iniciador de la elección es
el único proceso vivo, en cuyo caso al enterarse de que está enviándose un mensaje a
sí mismo no espera recibir un \acr{ACK}.

\section{Elección de líder}

Todos los procesos ejecutan esta rutina. Aquí es donde se lleva a cabo el
protocolo definido por el enunciado. Para satisfacer lo pedido se tiene un ciclo
que se ejecuta hasta que se termine el tiempo máximo para la definición de
líder.

En el mismo se lleva a cabo el siguiente mecanismo. Primero se consulta mediante
\texttt{MPI\_Iprobe} con tag \texttt{TAG\_ELECCION\_TOKEN} si se recibió
un token. En caso de recibir un mensaje el mismo es leído con \texttt{MPI\_Irecv} donde
luego, tomando el valor del emisor mediante \texttt{status.MPI\_SOURCE}, se le envía
el respectivo \acr{ACK}. Es entonces cuando se procede a analizar el token
recibido y actuar en base a su valor.

Si el proceso actual es el iniciador del mensaje recibido entonces existen dos
posibilidades: el mismo sigue siendo el candidato o este tomó otro valor. Si
sigue siendo el candidato entonces pasa a actualizar su estado a \texttt{LIDER},
de otra forma remplaza al iniciador por el candidato actual.

Si no es el iniciador pero tiene un número de \texttt{pid} mayor al del
candidato actual, entonces lo remplaza por el suyo.

Habiendo tomado la decisión correspondiente en función de los valores del token
queda enviar el mensaje al siguiente nodo. Al igual que con la rutina
\texttt{iniciar\_elección}, es necesario esperar un \acr{ACK} del destinatario
para corroborar que el mismo no esté muerto. Nuevamente el procedimiento
comienza enviando al \texttt{pid} señalado por \texttt{siguiente\_pid} mediante
\texttt{MPI\_Isend}. Luego queda a la espera consultando con
\texttt{MPI\_Iprobe} la llegada del \acr{ACK}. Si finaliza el tiempo de
espera sin haber recibido respuesta, se repite el procedimiento aumentando en
uno el valor del \texttt{pid} de destino.

\section{Problemas encontrados}

Finalizada la implementación del algoritmo, se procedió a testearlo con
diferentes casos. Al hacer esto, se descubrió que, como consecuencia de la
simplicidad con la que se implementó el protocolo de comunicación entre los
procesos, aparecían inconvenientes en diversas instancias del problema. A
continuación se describe un escenario problemático sencillo para ejemplificar
los problemas detectados.

Se considera un anillo donde solo hay dos procesos vivos, $P_1$ y $P_2$,
teniendo $P_1$ un \texttt{pid} considerablemente mayor que $1$. Si tanto $P_1$
como $P_2$ deciden iniciar una elección, podría pasar lo siguiente: $P_2$, que
sabe que es el último proceso, intenta enviar el token al proceso con
\texttt{pid} $1$, y queda esperando recibir un \acr{ACK}, que no llega porque
el proceso en cuestión está muerto. Cuando se cumple el tiempo previsto por el
protocolo, intenta con el proceso $2$, que también está muerto, y así
sucesivamente. Mientras $P_2$ está ocupado esperando un \acr{ACK}, $P_1$
intenta enviarle el token de su elección; $P_2$, que está ocupado esperando
que un proceso posterior, no responde con un \acr{ACK} y $P_1$ piensa que está
muerto, por lo que continúa con el protocolo con los procesos siguientes. Esto
quiere decir que $P_1$ nunca obtendrá respuesta de $P_2$, por lo que esta
elección se perderá. Peor aún, si el tiempo total de la etapa de elección es
suficiente, podría pasar que $P_1$ intente enviar un mensaje a un proceso con
un \texttt{pid} inexistente, provocando un error de \acr{MPI} que ocasiona el
fin de la ejecución del programa.

El problema expuesto es fácil de reproducir, ya que resulta relativamente
sencillo encontrar instancias que provoquen situaciones similares. La
principal causa de este comportamiento es el hecho de que durante la etapa de
envío de un token, los procesos quedan haciendo \emph{busy waiting} hasta
recibir el \acr{ACK} previsto por el protocolo, impidiendo que puedan
responder mensajes enviados por otros procesos.

\subsection{Posibles soluciones propuestas}

Se estudiaron varias soluciones posibles a este inconveniente. Primero se
pensó en alternar la espera de \acr{ACK} con el chequeo por la recepción de
nuevos tokens. De esta forma, al recibir un token, se puede responder con un
\acr{ACK} de forma inmediata. La complicación que surge de esto es que al
recibir un nuevo token este debe pasarse al proceso siguiente, debiendo
ahora esperar la recepción de un segundo \acr{ACK}. Esto se vuelve aún más
complejo si se considera que pueden seguir acumulándose los mensajes
recibidos, debiendo hacerse \emph{polling} por un número creciente de \acr
{ACK}.

Una solución alternativa es responder a los tokens recibidos con un \acr{ACK}
de forma inmediata, pero retrasar el envío de estos tokens hasta que se haya
recibido el \acr{ACK} por el cual se está esperando, evitando que se solapen
las esperas por más de un \acr{ACK}. El problema de esta solución es que debe
agregarse una estructura de datos (por ejemplo, una cola) para poder
almacenar los tokens cuyo envío está pendiente. Además, puede demorar de forma
considerable el progreso de la elección, provocando que el tiempo disponible
se termine antes de que esta haya finalizado.

Otra alternativa que se estudió, en una línea similar a la idea anterior, es
la de responder con un \acr{ACK} de forma inmediata pero, en lugar de
almacenar todos los tokens recibidos para su posterior envío, agregar lógica
que permita seleccionar y enviar solo uno de ellos, de una forma que no altere
el desarrollo de la elección de líder. Sin embargo, esta alternativa no solo
presenta inconvenientes similares a los planteados para el caso anterior, sino
que además requiere modificar el protocolo especificado de forma considerable.

\end{document}
